%% All credit goes to http://tex.stackexchange.com/users/48028/malipivo
%% Who originally posted this to stack overflow:
%% http://tex.stackexchange.com/questions/171500/how-to-print-a0-poster-as-an-array-of-combinable-a4-pages

\batchmode % Rather silent mode...

% Basic settings for user...
\def\malfile{input-A0.pdf} % name of the file
\def\malpage{1} % page in pdf file to be cut
\def\xmal{2} % number of cuts (x-axis), >1
\def\ymal{2} % number of cuts (y-axis), >1
\def\malscale{0.92} % postcut scaling (1.0 = no change; 0.92 common change)

% This file will be processed...
% Sort of hacking before \documentclass...
\makeatletter\input{graphicx.sty}\makeatother
\newbox\malbox
\setbox\malbox=\hbox{\includegraphics[page=\malpage]{\malfile}}
\ifnum\wd\malbox<\ht\malbox
  \def\malpaper{portrait}
\else
  \def\malpaper{landscape}
\fi
\documentclass[a3paper,\malpaper]{article}
\pagestyle{empty}
% Want to change the size? Uncomment and modify the following two lines...
%\paperwidth=21mm%
%\paperheight=29.7mm%
\pdfpagewidth=\paperwidth
\pdfpageheight=\paperheight
\usepackage{tikz}
% I am setting these two parameters right now before I forget to do that.
\tikzset{inner sep=0pt, outer sep=0pt}
\usetikzlibrary{calc}
% For intermediate+ TeXists... we can set a lot of things here...
%\tikzset{pagestyle/.style={draw=blue,line width=0.01\paperwidth}}% or draw=none, draw=blue; common setting is draw=none
\tikzset{pagestyle/.style={draw=none,line width=0.01\paperwidth}}% or draw=none, draw=blue; common setting is draw=none
% opacity=50, it is not needed as rectangle is drawn first

\begin{document}
% ...Setting dimensions part...
% Dimensions related to the x-axis; \paperwidth (in pt)...
\pgfmathparse{\wd\malbox/\xmal}
\let\realx=\pgfmathresult
\pgfmathparse{2*\realx}
\let\secondx=\pgfmathresult
\pgfmathparse{\wd\malbox}
\let\lastinx=\pgfmathresult
% Dimensions related to the y-axis; \paperheight (in pt)...
\pgfmathparse{\ht\malbox/\ymal}
\let\realy=\pgfmathresult
\pgfmathparse{\ht\malbox-\realy}
\let\beforelasty=\pgfmathresult
\pgfmathparse{\ht\malbox}
\let\lastiny=\pgfmathresult

% ...The main part...
% Cycle from top to bottom...
\foreach \y in {\lastiny,\beforelasty,...,\realy} {%
% Cycle from left to right side...
  \foreach \x in {\realx,\secondx,...,\lastinx} {%
    % The core of this TeX file...
    \newpage % One cut shown on one page...
    % Computation of the opposite corner...
    \pgfmathparse{\x-\realx}
    \let\previousx=\pgfmathresult
    \pgfmathparse{\y-\realy}
    \let\previousy=\pgfmathresult
    % Page number is... \thepage, therefore none computation is needed... :-)
    % Message to the terminal...
    \scrollmode % Write me something...
    \message{Processing page \thepage: \previousx, \previousy, \x, \y}%
    \batchmode % And go back to the silent mode...
    % Show me that part of the pdf file, give me some fancy drawing...
    % First: go to the left upper corner and then jump in the center of the page...
    \pgfmathparse{1in+\oddsidemargin+\parindent-0.5\paperwidth}% left-right direction
    \let\movemex=\pgfmathresult
    \pgfmathparse{1in+\topmargin+\headheight+\headsep+\the\fontdimen6\font-0.5\paperheight}%
    % up-down direction; or \baselineskip-2pt
    \let\movemey=\pgfmathresult
    % For more information and experiments of yours, please see the layouts package:
    % http://ctan.org/pkg/layouts
    % Including a portion of the pdf file into this document...
    \begin{tikzpicture}[overlay] %no need for: remember picture ;-)
    \node[scale=\malscale,pagestyle] at (-\movemex pt, \movemey pt) {%
      \includegraphics[page={\malpage}, viewport={\previousx pt} {\previousy pt} {\x pt} {\y pt}, clip, width=\paperwidth]{\malfile}%
      };% End of \node...
    \end{tikzpicture}%
  }% End of \x...
}% End of \y...
\end{document}
